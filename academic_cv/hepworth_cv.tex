%%%%%%%%%%%%%%%%%%%%%%%%%%%%%%%%%%%%%%%%%
% "ModernCV" CV and Cover Letter
% LaTeX Template
% Version 1.3 (29/10/16)
%
% This template has been downloaded from:
% http://www.LaTeXTemplates.com
%
% Original author:
% Xavier Danaux (xdanaux@gmail.com) with modifications by:
% Vel (vel@latextemplates.com)
%
% License:
% CC BY-NC-SA 3.0 (http://creativecommons.org/licenses/by-nc-sa/3.0/)
%
% Important note:
% This template requires the moderncv.cls and .sty files to be in the same
% directory as this .tex file. These files provide the resume style and themes
% used for structuring the document.
%
% \cventry{years}{degree/job title}{institution/employer}{localization}{grade}{description}
% http://theprofessorisin.com/2016/08/19/dr-karens-rules-of-the-academic-cv/
%%%%%%%%%%%%%%%%%%%%%%%%%%%%%%%%%%%%%%%%%

%----------------------------------------------------------------------------------------
%	PACKAGES AND OTHER DOCUMENT CONFIGURATIONS
%----------------------------------------------------------------------------------------

\documentclass[11pt,roman]{moderncv} % Font sizes: 10, 11, or 12; paper sizes: a4paper, letterpaper, a5paper, legalpaper, executivepaper or landscape; font families: sans or roman

\moderncvstyle{academic} % CV theme - options include: 'academic', 'casual' (default), 'classic', 'oldstyle' and 'banking'
\moderncvcolor{black} % CV color - options include: 'blue' (default), 'orange', 'green', 'red', 'purple', 'grey' and 'black'
\usepackage{hyperref}
\hypersetup{
    colorlinks=true,
    linkcolor=blue,
    filecolor=magenta,      
    urlcolor=cyan,
    pdftitle={Overleaf Example},
    pdfpagemode=FullScreen,
    }
\usepackage[scale=0.75]{geometry} % Reduce document margins
\usepackage{xhfill}
%\setlength{\hintscolumnwidth}{3cm} % Uncomment to change the width of the dates column
\setlength{\makecvtitlenamewidth}{10cm} % For the 'classic' style, uncomment to adjust the width of the space allocated to your name

%----------------------------------------------------------------------------------------
%	NAME AND CONTACT INFORMATION SECTION
%----------------------------------------------------------------------------------------

\firstname{Thomas} % Your first name
\familyname{Hepworth} % Your last name

% All information in this block is optional, comment out any lines you don't need
\title{Curriculum Vitae}
% \department{Physical Sciences Division}
\institution{The University of Winnipeg}
\address{18 Parkwater Crescent}{Winnipeg, MB, R2C 4W6} 
\mobile{431-998-0243}
\email{thomashepworth12@gmail.com}


%\extrainfo{additional information}

%----------------------------------------------------------------------------------------

\begin{document}

%----------------------------------------------------------------------------------------
%	COVER LETTER
%----------------------------------------------------------------------------------------

% To remove the cover letter, comment out this entire block

%\clearpage

%\recipient{HR Department}{Corporation\\123 Pleasant Lane\\12345 City, State} % Letter recipient
%\date{\today} % Letter date
%\opening{Dear Sir or Madam,} % Opening greeting
%\closing{Sincerely yours,} % Closing phrase
%\enclosure[Attached]{curriculum vit\ae{}} % List of enclosed documents

%\makelettertitle % Print letter title

%\lipsum[1-2] % Dummy text
%\lipsum[4] % Dummy text

%\makeletterclosing % Print letter signature

%\newpage

%----------------------------------------------------------------------------------------
%	CURRICULUM VITAE
%----------------------------------------------------------------------------------------

\makecvtitle % Print the CV title

%----------------------------------------------------------------------------------------
%	EDUCATION SECTION
%----------------------------------------------------------------------------------------
\href{https://www.linkedin.com/in/thomas-hepworth/}{LinkedIn Profile},
\href{https://github.com/thepworth3?tab=repositories}{GitHub Profile}
\section{Overview}
 I am a fourth-year physics honours student at the University of Winnipeg. I have spent my past three summers gaining research experience and contributing to the internationally recognized physics collaboration TUCAN at the University of Winnipeg and TRIUMF, Canada's particle accelerator centre. TUCAN is the TRIUMF Ultra Cold Advanced Neutron collaboration, and I have worked in the magnetics and neutron guide coating sectors of their neutron electric dipole moment (nEDM) experiment. I plan to complete an undergraduate thesis under TUCAN this upcoming year before graduating and pursuing a Ph.D in nuclear/particle physics. I have also been selected as one of six Canadian Institute of Particle Physics (IPP) summer student fellows and will be spending the second half of the summer of 2024 working on the Baryon Antibaryon Symmetry Experiment (BASE) at CERN. I am passionate about science outreach and work and volunteer in leadership positions to encourage science enrollment by young Canadians in my free time. 

\section{Education}

\cvjob[year=2021-2025,
       position=B.Sc Honours (Physics)  4.491/4.5 Major GPA, 
       company=University of Winnipeg , ]{}
       
\cvjob[year=,position = 4.477/4.5 Cumulative GPA, 
       company= , ]{}


%----------------------------------------------------------------------------------------
%	EMPLOYMENT SECTION
%----------------------------------------------------------------------------------------

\section{Research Employment}

\cvjob[year=2024,
       position= Summer Student Programme,
       company=  CERN]{}
\cvjob[year=2024,
       position=NSERC USRA Research Student,
       company=The University of Winnipeg/TRIUMF]{}
       
\cvjob[year=2023,
       position=NSERC USRA Research Student,
       company=The University of Winnipeg/TRIUMF]{}
\cvjob[year=2022,
       position=Research Assistant,
       company=The University of Winnipeg/TRIUMF]{}


%----------------------------------------------------------------------------------------
%	AWARDS SECTION
%----------------------------------------------------------------------------------------
\section{AWARDS}
\cventry{2025-2029}{Max Planck Institute of Nuclear Physics (MPIK) International Max Planck Research School (IMPRS) for Precision Tests of Fundamental Symmetries (PTFS) PhD fellowship}{University of Heidelberg}{}{}{One of the most prestigious PhD fellowships in Nuclear physics in the world which will support my studies at the University of Heidelberg and the 
Institut Laue-Langevin (ILL) for the PanEDM experiment.}


\cventry{2024}{Institute for Particle Physics (IPP) Summer Student Fellowship (approx \$7000)}{one of six Canadian students selected by the IPP to participate in CERN's prestigious summer student program}{}{}{}

\cventry{2024}{\mdseries \textbf{Canadian Association of Physicists Department of Nuclear Physics Best Student Presentation Competition 3$^{rd}$ place winner}}{Placed 3$^{rd}$ in the CAP DNP student talk competition as an undergrad in a competition comprised of mostly graduate students}{}{}{}

\cventry{2024}{\mdseries Donald Kerr Scholarship (\$2511)}{Awarded to the top graduating physics student at the University of Winnipeg}{}{}{}


\cventry{2024}{\mdseries University of Winnipeg Academic Proficiency Scholarship (\$500)}{}{}{}{}

\cventry{2024}{\mdseries H.V. Rutherford Scholarship (\$3000) }{Award to a student who expresses interest and talent for teaching at the university level}{}{}{}


\cventry{2024}{\mdseries Chancellor W. John A. Bulman Scholarship (\$4337)}{}{}{}{}

\cventry{2024}{\mdseries University of Winnipeg Undergraduate Student Research Travel Grant (\$750)}{Awarded to supplement costs of travel to the Japan Proton Accelerator Research Complex (JPARC)}{}{}{}

\cventry{2024}{Canadian Institute of Nuclear Physics Undergraduate Research Scholarship (CINP URS) \mdseries (\$6800)}{awarded to enable gifted undergraduates to work with a supervisor on nuclear physics research in Canada}{award was declined}{}{}

\cventry{2024}{\mdseries Natural Sciences and Engineering Research Council Undergraduate Student Research Award (NSERC USRA) (\$6000)}{awarded to provide outstanding students with opportunities in science and engineering research}{}{}{}

\cventry{2024}{\mdseries University of Winnipeg Research Office Photo competition}{\href{https://news.uwinnipeg.ca/research-photography-competition-delivers-wow-factor/}{Category Winner} (\$300)}{}{}{}



\cventry{2024}{\mdseries David R. Dyck Prize in History (\$1218)}{awarded for the best essay written in the history of science, technology, or medicine. Essay title: How James Clerk Maxwell’s Relationship With Michael Faraday, and Maxwell’s Religious Perspectives Shaped His Electromagnetic Field Theory}{}{}{}

\cventry{2023}{\mdseries Dr. Randy Kobes Memorial Scholarship (\$1000)}{awarded to a student in the faculty of science who demonstrates academic excellence and has made significant contributions to the community, especially in the area of scientific outreach}{}{}{}
\cventry{2023}{\mdseries Brian J. Hyslop Memorial Scholarship in Physics (\$2000)}{awarded to a student who demonstrates both passion and ability to perform productive research, and is likely to contribute to society}{}{}{}
\cventry{2023}{\mdseries B. G. Hogg Scholarship in Physics (\$541)}{awarded to outstanding physics major to attend a physics conference}{}{}{}
\cventry{2023}{\mdseries University of Winnipeg Academic Proficiency Scholarship, Student of Highest Distinction (\$400)}{awarded to students with GPAs exceeding 4.0}{}{}{}
\cventry{2023}{\mdseries Natural Sciences and Engineering Research Council Undergraduate Student Research Award (NSERC USRA) (\$6000)}{awarded to provide outstanding students with opportunities in science and engineering research}{}{}{}
\cventry{2022}{The University of Winnipeg General Bursary (\$1000)}{}{}{}{}
\cventry{2022}{Duckworth Scholarship in Physics (\$1000)}{awarded to physics major of outstanding academic promise}{}{}{}
\cventry{2022}{Transcona Collegiate Alumni Scholarship (\$500)}{awarded to an outstanding student from the previous graduating class of Transcona Collegiate}{}{}{}
\cventry{2021}{Royal Canadian Legion Transcona Branch \#7 Poppy Trust Fund Award (\$500)}{}{}{}{}
\cventry{2021}{Stella Wujek Scholarship for Science (\$1000)}{awarded to student demonstrating academic excellence in the sciences}{}{}{}
\cventry{2021}{River East Transcona School Division Medal Award (\$1500)}{awarded to the top graduating student of Transcona Collegiate}{}{}{}
\cventry{2021}{Helen Shandruk Memorial Award (\$2000)}{awarded to a top graduating student of Transcona Collegiate}{}{}{}
\cventry{2021}{The University of Winnipeg Special Entrance Scholarship (\$2250)}{awarded to student with high school grade average greater than 95\%}{}{}{}
\cventry{2021}{Casera Credit Union Memorial Bursary (\$750)}{}{}{}{}


\section{Physics Experiences}
\cventry{2024-2025}{\mdseries TUCAN (TRIUMF Ultra Cold Advanced Neutron Collaboration }{}{}{}{}
\cventry{}{\mdseries Currently I am completing a bachelor's thesis in the Neutron Guide Coating Facility at the University of Winnipeg. The facility employs pulsed laser ablation and e-beam evaporation to apply thin film coatings onto the inside of Ultracold Neutron (UCN) transport tubes and storage vessels. I am completing Monte Carlo simulations to predict the material properties of these UCN guides. In January, I will travel to the Japan Proton Accelerator Research Complex to test these guides at their UCN source. My results will be published, and used for future simulations of the UCN experiment at TRIUMF. This project involves advanced 3D modelling of experimental geometries and coding and data analysis for my simulations. Work ongoing}{}{}{}{}

\cventry{2024}{\mdseries CERN - Baryon Antibaryon Symmetry Experiment}{}{}{}{}
\cventry{}{\mdseries The BASE collaboration is specialized in the use of advanced Penning trap setups as well as cryogenic superconducting detection systems with single particle resolution to perform world-leading
measurements on protons and antiprotons. Past measurements include the comparison of antiproton and proton charge-to-mass ratio with a fractional precision of 16 parts per trillion (ppt) and
the antiproton g-factor with a fractional precision of 1.5 parts per billion (ppb). I am working on the development of superconducting electric joints to allow for better performance of BASE superconducting coil systems.}{}{}{}{}

\newpage
\cventry{2022-2024}{\mdseries TUCAN (TRIUMF Ultra Cold Advanced Neutron Collaboration)}{}{}{}{}
\cventry{}{\mdseries TUCAN is an international collaboration of physicists from Canada, the United
States, and Japan. They have received over \$30 million in Canadian and Japanese funding
since 2009. TUCAN plans to measure the neutron electric dipole moment
(nEDM) with an uncertainty of 10$^{-27}$ecm, a level of precision an order of magnitude more precise than current
generation nEDM experiments. Measuring the nEDM probes new physics beyond
the standard model (BSM). More specifically, a new precise measurement can help investigate
the baryon asymmetry problem, the matter-antimatter imbalance in the universe. I have spent the past two summers working for the TUCAN
collaboration. I have been involved in multiple projects, mostly focusing on precision magnetic field control and characterization of the nEDM experiment.
I was responsible for the characterization of the magnetically shielded room (MSR) during its construction. I wrote codes to analyze, simulate, and collect data related to the MSR. I also characterized magnetic sensors that were used in the MSR. In addition, I characterized a new neutron detector before its testing with neutrons in Japan, and I designed components to adapt to the beam-line at the Japan Proton Accelerator Research Complex (JPARC). I traveled to TRIUMF in Vancouver British Columbia to complete this research throughout the past two summers. A publication is expected to arise from the magnetic field research of which I will be a co-author. In 2024 I transitioned to working on the UCN source at TRIUMF. I have taken several shifts supervising our cool-down, and I analyzed cryostat data in Python after identifying issues worth investigating further during my shifts. Work ongoing}{}{}{}{}
\cventry{2023}{\mdseries University of Winnipeg Neutron Guide Coating facility}{}{}{}{}
\cventry{}{\mdseries The Neutron Guide Coating Facility employs pulsed laser ablation and ebeam evaporation to apply thin film coatings onto the inside of UCN transport tubes and storage vessels. The facility specializes in hydrogen-free diamond-like carbon. I helped arrange the lasers and vacuum chambers into their final alignments, installed compressed cylinder gas and water lines for the equipment, and assisted in excimer laser testing. This work was done in conjunction with my research with TUCAN and was supported by an NSERC USRA. A publication is expected to arise from this work in the winter of 2024, and I will be continuing this work in an undergraduate thesis}{}{}{}{}


%----------------------------------------------------------------------------------------
%	PUBLICATIONS SECTION
%----------------------------------------------------------------------------------------
% \section{PUBLICATIONS}

% % First author
% awaiting input from Russ


% %----------------------------------------------------------------------------------------
% %	MANUSCRIPTS SECTION
% %----------------------------------------------------------------------------------------
% \section{MANUSCRIPTS in preparation}
% awaiting input from Russ

% %-------------------------------------------------------------------------------
% %	CONFERENCES SECTION
% %-------------------------------------------------------------------------------
\section{PRESENTATIONS and Articles}

\cvconf[year=2024,
	   title= Traveling the World Doing Research,
	   conference= University of Winnipeg Research Week,
	   style=Invited talk]{}

\cvconf[year=2024,
	   title= Experiences in Undergraduate Research,
	   conference= University of Winnipeg Research Week,
	   style=Invited talk]{}

\cvconf[year=2024,
	   title= Superconducting Joints for the BASE Superconducting Coil System ,
	   conference= 19$^{th}$ Randy Kobes Poster Symposium (Placed 3$^{rd}$),
	   style=Poster]{}
    
\cvconf[year = 2024, title = \href{https://news.uwinnipeg.ca/pump-up-the-neutrons/}{Pump up the Neutrons}, conference = University of Winnipeg News, style =Article]{}
\cvconf[year = 2024, title = \href{https://news.uwinnipeg.ca/physics-student-off-to-switzerland/}{Physics Student off to Switzerland}, conference = University of Winnipeg News, style =Article]{}
\cvconf[year=2024,
	   title=Superconducting Joints for the BASE Superconducting Coil System,
	   conference= CERN Summer Student Talks,
	   style=Contributed talk]{}

\cvconf[year=2024,
	   title=Measurements of a Magnetically Shielded Room for a Neutron EDM Experiment,
	   conference= Canadian astroparticle summer student talk competition,
	   style=Contributed talk]{}
\cvconf[year=2024,
	   title=Measurements of a Magnetically Shielded Room for a Neutron EDM Experiment,
	   conference= Canadian Association of Physicists Congress,
	   style=Contributed talk - 3rd place winner in nuclear physics oral competition]{}
\cvconf[year=2023,
	   title=Experiences in Undergraduate Research,
	   conference= \href{https://www.uwinnipeg.ca/physics/news/2023/09/celebrating-excellence-in-undergraduate-research.html}{Research Week} - The University of Winnipeg,
	   style=invited panelist]{}
\cvconf[year=2023,
	   title= Summer student research opportunities,
	   conference= University of Winnipeg physics colloquium series ,
	   style=Contributed talk]{}
\cvconf[year=2023,
	   title=Simulations of Magnetic Fields Inside a Magnetically Shielded Room for the TUCAN nEDM Experiment,
	   conference= University of Winnipeg NSERC USRA event,
	   style=Poster]{}

\cvconf[year=2023,
	   title=Simulations of Magnetic Fields Inside a Magnetically Shielded Room for the TUCAN nEDM Experiment,
	   conference= 18$^{th}$ Randy Kobes Poster Symposium,
	   style=Poster]{}
\cvconf[year=2023,
	   title=Simulations of Magnetic Fields Inside a Magnetically Shielded Room for the TUCAN nEDM Experiment,
	   conference= Canadian astroparticle summer student talk competition,
	   style=Contributed talk]{}
\cvconf[year=2023,
	   title=\href{https://www.uwinnipeg.ca/physics/news/2023/09/celebrating-excellence-in-undergraduate-research.html}{Celebrating excellence in undergraduate research},
	   conference= University of Winnipeg News ,
	   style=Article]{}
\cvconf[year=2022,
	   title= Summer student research opportunities,
	   conference= University of Winnipeg physics colloquium series ,
	   style=Contributed talk]{}
\cvconf[year=2022,
	   title=Characterization of Optically Pumped Total Field Magnetometers,
	   conference= 17$^{th}$ Randy Kobes Poster Symposium,
	   style=Poster]{}





%-------------------------------------------------------------------------------
%	TEACHING SECTION
%-------------------------------------------------------------------------------
\section{TEACHING}
\cvdoubleitem{2022-Ongoing}{Head Teaching Assistant and Marker}{}{Foundations of Physics}{}{}{}

\cvdoubleitem{2024-2025}{Marker}{}{Foundations of Physics Lab}{}{}{}

\cvdoubleitem{2024}{Marker}{}{Math Physics 1}{}{}{}

\cvdoubleitem{2024}{Head Teaching Assistant and Lab Marker}{}{Optics and Waves}{}{}{}

\cvdoubleitem{2024}{Lab material development}{}{Optics and Waves}{}{}{}


\cvdoubleitem{2022-Ongoing}{Exam Invigilator}{}{Foundations of Physics, Introduction to Physics, Astronomy, Concepts in Science}{}{}{}

\cvdoubleitem{2022-2023}{Head Teaching Assistant and Marker}{}{Introduction to Physics Lab}{}{}{}

\cvdoubleitem{2023}{Marker and Teaching Assistant}{}{Electricity and Magnetism}{}{}{}

\cvdoubleitem{2023}{Marker and Teaching Assistant}{}{Electricity and Magnetism Lab}{}{}{}

\cvdoubleitem{2023}{Teaching Assistant/Instructor}{}{High school enrichment Astronomy course}{}{}{}



%-------------------------------------------------------------------------------
%	COMPLEMENTARY EDUCATION SECTION
%-------------------------------------------------------------------------------

%\section{COMPLEMENTARY EDUCATION}
%\cventry{2023}{Crane Operator Training}{}{}{}{TRIUMF}

%-------------------------------------------------------------------------------
%	UNIVERSITY SERVICE SECTION
%-------------------------------------------------------------------------------
\section{Science Outreach and other volunteering}
\cventry{2024-2025}{President, University of Winnipeg Physics Student Association (UWPSA)}{}{}{}{}
\cventry{2024, 2025}{Presented about magnets for third grade class at Joseph Teres Elementary School. 5 classes total to date}{}{}{}{}

\cventry{2022-2023}{Wii Chiiwaakanak Learning Centre STEM camp, science camp to encourage indigenous enrollment in science. Uwinnipeg won the \href{https://www.sciencerendezvous.ca/hall_of_fame/steam-big-award/2023-steam-big-award/}{2023 STEAM Big! Award}}{}{}{}{}
\cventry{2022-2023}{University of Winnipeg Future Student Night, represented the physics department to encourage physics enrollment at the University of Winnipeg}{}{}{}{}
\cventry{2023-2024}{University of Winnipeg Open House,represented the physics department to encourage physics enrollment at the University of Winnipeg}{}{}{}{}
\cventry{2023}{High school physics presentations, travelled to Winnipeg high schools to  encourage physics enrollment at the University of Winnipeg on behalf of the physics department}{}{}{}{}
\cventry{2023-2024}{Treasurer, University of Winnipeg Physics Student Association (UWPSA)}{}{}{}{}
\cventry{2018-Ongoing}{Tutoring, casually tutored middle and high school students in science and math classes. Also have done tutoring at the University level, some of which is paid work and some as a volunteer}{}{}{}{}
\cventry{2015-2020}{Flag Football Coach for an after school program at Joseph Teres School. This involved going to a tournament each year with the students. Program was shutdown in this form in 2020}{}{}{}{}


% %-------------------------------------------------------------------------------
% %	RELATED WORK SECTION
% %-------------------------------------------------------------------------------
% \section{related work}
% \cventry{Software}{\mdseries$\beta$-NMR and $\beta$-NQR data fitting and visualization GUI and API}{}{}{}{\url{https://pypi.org/project/bfit/}}

% This section of realted work maybe shouldn't be on a scholarship cv

%-------------------------------------------------------------------------------
%	SKILLS SECTION
%-------------------------------------------------------------------------------
\section{SKILLS}
\cventry{\small{Languages}}{\mdseries English (native)}{}{}{}{Python, \LaTeX, LabView}
\cventry{\small{Certifications}}{\mdseries Canadian Nuclear Energy Worker (NEW)}{}{}{}{}
\cventry{\small{Experimental}}{\mdseries Magnetics, Magnetometers, experimental design, Vacuum systems, pressurized gas handling, Swagelok, fine tip soldering}{}{}{}{}
\cventry{\small{Computational}}{\mdseries Magnetic simulations in python, NumPy, Pandas, remote instrument control}{}{}{}{}
\cventry{\small{Engineering}}{\mdseries CAD, 3D printing}{}{}{}{}
\cventry{\small{Teaching}}{\mdseries Tutoring, marking, undergraduate lab demonstration, study skills coaching, lab course material development}{}{}{}{}
\cventry{\small{Soft Skills}}{\mdseries Leadership, organization, communication, presentations, time management, task delegation}{}{}{}{}
\cventry{\small{Other Skills}}{\mdseries Using industrial and machining equipment, soldering, tool and assembly skills}{}{}{}{}


%\cventry{}{}{}{}{}{}{}

%\renewcommand{\listitemsymbol}{-~} % Changes the symbol used for lists

%\cvlistdoubleitem{Piano}{Chess}
%\cvlistdoubleitem{Cooking}{Dancing}
%\cvlistitem{Running}

%----------------------------------------------------------------------------------------

\end{document}
