\documentclass{simplehipstercv_cl}

\usepackage{siunitx}
\usepackage[unicode, colorlinks=true, allcolors=linkcolor]{hyperref}

% Custom Commands
\newcommand{\bnmr}{$\beta$-NMR}
\newcommand{\bnqr}{$\beta$-NQR}
\newcommand{\elip}{$^8$Li$^+$}
\newcommand{\eli}{$^8$Li}
\newcommand{\lip}{Li$^+$}

\textheight 9.6in

\name{Thomas}{Hepworth}
\tagline{Letter of Introduction - Physicist at Ideon}
\socialinfo{
	\linkedin{}
	\smartphone{}
	\email{}
    \github{}\\
	\address{}
	\infos{Canadian Citizen}
}

\newcommand{\sepa}{0.75cm}

\begin{document}

	\makecvheader

	\makecvfooter
		{\textsc{ \selectlanguage{english}\today}}
				{%\thepage
		}{\textsc{Thomas Hepworth - Letter of introduction}}

\vspace{0.75cm}
%headline

\color{accentcolor}
\today \par \vspace{-0.1cm}
\vspace{0.5cm}


% Opening
% Tell them very simply and succinctly: Who you are professionally, what you can do for them, why you are interested in the job and/or employer. If there is some recent event or success the employer had that you can incorporate into why you are interested in the job, this can be very compelling. Limit to 3-4 sentences at most.

% The second paragraph. If you have identified either through your conversations with the hiring supervisor or a careful read of the job description what the most critical duty or qualification is for this hire, then make this the subject of the second paragraph. How will you meet this need?
As the principal investigator in three \bnmr\ experiments (M1760, M1892, and M2072), the scientific program which I developed in our group was focused on surface measurements in amorphous materials, and includes a new collaboration with a nano-lithography group from Ireland.

% Paragraphs 3, possibly 4. Point them to the evidence in your resume that you have the experience to get the major duties of the job done. If you can cover it in just one paragraph, then don’t add a fourth. If there are two broad areas (e.g., data analysis and reporting or grant writing and project implementation), then making each area the subject of the each paragraph is reasonable, but keep them short.


% Closing paragraph. Keep this very short, 2-3 sentences. If you have nothing more to cover that wasn’t in the previous paragraphs, then simply say how it would be a pleasure to join their team and you look forward to learning more about the position and their organization. Close with “Sincerely,” (or similar) and then type your name. Do not print, sign, and scan—the employer needs to be able to do a keyword search on your letter and that is impossible with a signed and scanned letter. They don’t need your hand written signature.

\vspace{0.5cm}
\raggedright
Sincerely,\\
\vspace{0.25cm}
Thomas Hepworth

\end{document}

