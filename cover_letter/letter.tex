\documentclass{simplehipstercv_cl}

\usepackage{siunitx}
\usepackage[unicode, colorlinks=true, allcolors=linkcolor]{hyperref}

% Custom Commands
\newcommand{\bnmr}{$\beta$-NMR}
\newcommand{\bnqr}{$\beta$-NQR}
\newcommand{\elip}{$^8$Li$^+$}
\newcommand{\eli}{$^8$Li}
\newcommand{\lip}{Li$^+$}

\textheight 9.6in

\name{Derek}{Fujimoto}
\tagline{Letter of Introduction - Physicist at Ideon}
\socialinfo{
	\linkedin{derek-fujimoto}
	\smartphone{(778) 873-0054}
	\email{dfujim@protonmail.com}
    \github{dfujim}\\
	\address{Vancouver, Canada}
	\infos{Canadian Citizen}
}

\newcommand{\sepa}{0.75cm}

\begin{document}

	\makecvheader

	\makecvfooter
		{\textsc{ \selectlanguage{english}\today}}
				{%\thepage
		}{\textsc{Derek Fujimoto - Letter of introduction}}

\vspace{0.75cm}
%headline

\color{accentcolor}
\today \par \vspace{-0.1cm}
\vspace{0.5cm}


% Opening
% Tell them very simply and succinctly: Who you are professionally, what you can do for them, why you are interested in the job and/or employer. If there is some recent event or success the employer had that you can incorporate into why you are interested in the job, this can be very compelling. Limit to 3-4 sentences at most.
I am a graduating student at the University of British Columbia, having recently defended my doctoral thesis earlier this summer. Throughout my studies, working with implanted-ion $\beta$-detected NMR (\bnmr) at TRIUMF, I have accumulated expertise working with UHV and cryogenic systems, the polarization and implantation of radioactive ion beams, as well as in-depth experience with the technical aspects of the \bnmr\ spectrometer, DAQ, and data analysis.

% The second paragraph. If you have identified either through your conversations with the hiring supervisor or a careful read of the job description what the most critical duty or qualification is for this hire, then make this the subject of the second paragraph. How will you meet this need?
My familiarity with the spectrometer and hardware is in large part due to the high-temperature upgrade which I proposed, implemented, and commissioned. The major design challenges of this work were accommodating the sample environment, which was under UHV, high magnetic field, and a wide range of temperatures (\SIrange{3}{400}{\K}); and the spatial restrictions posed by the coldfinger cryostat.

I also took the initiative to design and write a number of software packages for both data analysis and beamspot imaging. These have been universally adopted by both local and visiting experimenters, and is capable of both rapid online work and sophisticated publication-quality analyses. These works have also been used in the development of the new \bnmr\ DAQ (launched this summer), and in the analysis of $\mu$SR experiments at TRIUMF.

As the principal investigator in three \bnmr\ experiments (M1760, M1892, and M2072), the scientific program which I developed in our group was focused on surface measurements in amorphous materials, and includes a new collaboration with a nano-lithography group from Ireland.

% Paragraphs 3, possibly 4. Point them to the evidence in your resume that you have the experience to get the major duties of the job done. If you can cover it in just one paragraph, then don’t add a fourth. If there are two broad areas (e.g., data analysis and reporting or grant writing and project implementation), then making each area the subject of the each paragraph is reasonable, but keep them short.
I am also adept at experimentation and study in smaller offline contexts. During my Masters, I characterized a PMT for the Belle II experiment at KEK, Japan, with the intent of determining its efficacy for use as a detector in the end cap calorimeter. These experiments involved the heavy use of NIM logic circuits and high-speed electronics, as well as regular virtual meetings with our Japanese colleagues.

Throughout my graduate studies I worked closely with the Physics Education Research group at UBC to develop course materials and train teaching assistants for a first year undergraduate lab. I was also involved in the design and piloting of the the instructor TA program in the physics department. As a result, I would be very comfortable supervising students.

% Closing paragraph. Keep this very short, 2-3 sentences. If you have nothing more to cover that wasn’t in the previous paragraphs, then simply say how it would be a pleasure to join their team and you look forward to learning more about the position and their organization. Close with “Sincerely,” (or similar) and then type your name. Do not print, sign, and scan—the employer needs to be able to do a keyword search on your letter and that is impossible with a signed and scanned letter. They don’t need your hand written signature.
It would be a pleasure to have the opportunity join your team and I am excited for the opportunity to apply my expertise to a new experiment.

\vspace{0.5cm}
\raggedright
Sincerely,\\
\vspace{0.25cm}
Derek Fujimoto

\end{document}

